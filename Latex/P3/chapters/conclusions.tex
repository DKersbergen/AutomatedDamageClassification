\chapter{Conclusions} \label{concl}
The main goal of this research was to find a method for the automatic classification of damage inflicted by hurricanes on the island of St. Maarten using remotely sensed data to support the operations of the \ac{nlrc} and other humanitarian \ac{ngo}. To achieve this goal the following question was defined, \textit{Is the use of remotely sensed data a viable option for the automatic classification of hurricane inflicted damage?} The various sub-questions all contributed to a stream-lined method for answering this question. 

\section{How is damage determined?}
To determine the damage to buildings, various information densities can be used. Section \ref{sec:clas} provides an exemplified overview of the three nominal gradations used in this research. These are the detection, classification, and assessment of damage. The difference between these methods is in the detail to which the damage is described. The increase in inclusion of detail is proportional to the ongoing developments within the \ac{drm} activity cycle of a humanitarian organisation after a disaster; the further into this development, the more detail is necessary for the correct planning of humanitarian aid. Limited standardised methods exist for all three nominal gradations of damage determination, this research therefore proposes the use of a classification diagram, based on efforts from the Harvard Humanitarian Initiative and 510 from the \ac{nlrc}. This classification diagram is applicable to both automated and manual approaches for damage classification and allows for clear distinction between damage classes while avoiding ambiguity.

\section{What criteria are set for damage classification methods?}
Operation criteria for the classification of damage are set by humanitarian organisations, like the \ac{nlrc}. These are based on the information needs in a disaster situation. As described in section \ref{sec:dis}, this research focusses on the emergency relief and rehabilitation phases. The detail requirements are set to block or building level, in which classification, as described in section \ref{sec:clas}, suffices. To be able to select a method for rapid assessment using remotely sensed data, a framework has been established with 4 parameters: [1] Accuracy, [2] Acquisition time, [3] Acquisition method, and [4] Resolution. These are the four parameters that should be taken into account in the development or assessment of a method for use in a humanitarian context. In this context, accuracy describes the percentage of correct classification, which can be defined in various manners, as described in section \ref{sec:interrater}. A clear guideline to the accuracy is to include all people in need of help, while still differentiating to ensure effective distribution of means. Acquisition time is an operation requirement that relates to the time constraints in a \ac{drm} activity cycle. For a method to be implemented in a specific part of the humanitarian response to a disaster, time constraints the possibilities for data collection. This is closely linked with the acquisition method, which describes how and what data can be collected. This would be limited by sources of data, as well as the specialities within a humanitarian team. Resolution is a two-fold parameter to describe methods for damage detection or classification. It describes both the resolution of the input data and output information, which are constraint by time and resources. In the various phases of the \ac{drm} activity cycle, the requirements for the information vary. The definition of sufficient detail in the output information, is proportionally linked with the activity of ongoing development. Furthermore, are humanitarian mission restricted by resources in the various phases to process high resolution data.\\

\noindent For most of these parameters no specific guidelines for values are provided, as every situation in humanitarian action is different and all require slightly different datasets or implementations. In the selection of method for damage classification, it is therefore inevitable that guidelines to these parameters are set by the specific humanitarian mission and will vary as a result.

\section{Which methods already exist?}
A broad selection of methods are already available in academic literature. Section \ref{sec:etech} provides a non-exhaustive overview. While all approaches differ in exact techniques used to determine damage, the range of remotely sensed data they use is limited to satellite optical and \ac{sar} data or \ac{uav} optical data. This is only a small subset of the possibilities described by \citet{Kerle2008}. Most quoted accuracies range between 70\% - 90\%, with acquisition times under a week. This indicates suitability for use in the \ac{drm} activity cycle after a disaster. \citet{Vetrivel2016b} and \citet{Yun2015} describe methods with most potential. The \ac{cnn} approach by \cite{Vetrivel2016b} yields the highest accuracy claims, with over 90\% accuracy. The equalisation and univariate differencing technique described by \cite{Yun2015} has found the most traction in the humanitarian sector with uses by various \ac{ngo}.

\section{How do these methods perform?}
To indicate performance, a pure definition of accuracy is not sufficient. In case of unbalanced data this might lead to a false indication of accuracy. Other inter-rater statistics are therefore necessary. In this research the Cohen Kappa coefficient and F1-score have been used for their inclusion of randomness in the accuracy measurement, and inclusion of precision and recall, respectively. These approaches to accuracy are used to provide a more complete overview of the performance of the methods. \\

\noindent In this research the methods by \citet{Vetrivel2016b} and \citet{Yun2015} have been adapted in compared. Both methods have been implemented with the data they were intended for \ac{uav} optical and satellite \ac{sar}, respectively. However, these methods have also been extended to allow for comparison between the \ac{sar} data and optical data, this translate to implementations for the other datasets as well. Due to characteristics of the \ac{sar} data, it is not possible to include an implementation based on a \ac{cnn} approach. Furthermore, all implementations have been extended to allow for smaller granularity in the output information, which translates to an implementation for damage detection and damage classification.\\

\noindent Quantified inspection of the methods indicate that the adapted approached of Equalisation and Subtraction and a \ac{cnn} approach are not sufficiently suitable for direct implementation in damage classification, with low values for both the Cohen Kappa coefficient and average F1-score. The methods based on differencing perform all roughly equal, while the \ac{cnn} approach is hindered by inappropriate input features. For the detection of damage these methods show suitable results for the use by humanitarian organisations. The methods based on optical image differencing work adequately, specifically on those features which indicate damage. The same approach based on the \ac{sar} data performs not as well, especially with regards to the identification of damage. The \ac{cnn} approach suffers the same problems as in the classification approach and perform not up to par.

\section{How does the state of the art compare to these methods?}
The state of the art for damage determination in humanitarian aid is the use of visual inspection by Copernicus and UNOSAT. The comparable information for this research is based upon satellite optical data. Compared to the ground truth data this approach over-estimates the damage in an area. This is in line with the approach of humanitarian organisations which prefer to supply more people with aid than have people left out. Comparatively this approach is quantifiable less accurate, however has some major implementation advantages. While the damage is overestimated, other humanitarian organisations are not burdened with the task of collecting and processing data. This set-up has a clear operational advantage as less specialised personnel is necessary for the implementation of new automated measures. It offers an all-inclusive solution with fast results. In comparison a subtraction approach based on \ac{uav} optical imagery would take more investment to obtain sufficient workable coverage of a larger scale disaster, even if it provides better accuracy. A similar problem arises with the use of \ac{cnn} approaches. In that case specialised people are necessary to train and maintain a \ac{cnn} to allow for fast implementation after a disaster, which can furthermore be hindered by the lack of granular data collection.

\section{Summary}
\noindent \textit{Is the use of remotely sensed data a viable option for the automatic classification of hurricane inflicted damage?} Yes, there are many possibilities for the use of remotely sensed data to be used for the classification of damage. However, the practical implementations are not available yet. Various possibilities for future research, to make this possible, have been illustrated in chapter \ref{reco}. Methods for damage detection exist and can be introduced in emergency situations, however they are not user friendly for the humanitarian delegates in the field, without technical know-how. The result of this is the return to practical solutions offered by other \ac{ngo}, based on visual interpretation. A more human approach to the problem would allow non-technicians to get acquainted with the possibilities of remotely sensed data and the benefits for disaster response. This could pave the path for willingness to invest in new automated approaches to make humanitarian aid more effective and efficient.
