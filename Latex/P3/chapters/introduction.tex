\chapter{Introduction} \label{intro}

\noindent Weather related natural disasters cost the world economy around 100 billion dollars every year \citep{Kousky2014}. According to the \citet{CRED2015} 69.800 deaths per year are inflicted as the result of these disasters and earthquakes. The effects are felt around the world, however most deaths occur in low or middle income areas. Advances in technology and preparedness have decreased the amount of deaths caused by natural disasters since the second part of the previous century \citep{UN2004}. However, due to an increase in the frequency of disasters [figure \ref{fig:graph1}] more people are affected and more damages occur; with the most economic damage recorded in 2011 \citep{Coppola2015, Kerle2015}. 2017 was no exceptions to both trends, as it was the year with the second most economic damage but with less people killed \citep{RE2018}.  \\
\begin{figure}[h]
	\begin{tikzpicture}
	\centering
	\begin{axis}[
	axis lines = left,
	width=0.95\textwidth,
	height=\axisdefaultheight,
	ymin=0,
	xlabel={year},
	ylabel={Annual disaster count},
	xticklabel style=
	{/pgf/number format/1000 sep=,rotate=45,anchor=east,font=\scriptsize},
	yticklabel style=
	{font=\scriptsize},
	]
	\addplot [newBlue, very thick] table[x=x,y=y,col sep=comma]{data/disaster.csv};
	\end{axis}
	\end{tikzpicture}
	\caption{\footnotesize{Natural disaster per year from 1965 to 2016 [From:  EM-DAT: The Emergency Events Database - Universite catholique de Louvain (UCL) - CRED, D. Guha-Sapir - www.emdat.be, Brussels, Belgium]}}
	\label{fig:graph1}
\end{figure}

\noindent One of the major disasters of 2017 was hurricane Irma, being labelled the worst storm in the Caribbean in recordARed history \citep{Daniell2017}. In the first week of September this hurricane raged over multiple islands causing billions worth of damage, affecting millions in its path \citep{Phipps2017,Daniell2017}. One of the islands affected is St. Maarten, part of the Kingdom of the Netherlands. It was hit by the eye of the hurricane on the 6th of September with winds up to 185 miles per hour \citep{Wilts2017}. Two other major hurricanes passed over the area in the weeks following hurricane Irma, however, fortunately these did little to no extra damage on the island \citep{Gray2017,Bijnsdorp2017}. First damage estimates show that 70 - 90\% of the island may be affected by the storm \citep{Rodekruis2017,UNOSAT2017}. The indication and location of the damage caused is a leading planning tool for organisations like the \ac{nlrc}, providing a first indication of the most vulnerable people in an affected area. Indications of damage have allowed the \ac{nlrc} to help 18.881 individual people since the landfall of hurricane Irma and subsequent relief operation; and long term operations are being started right now to facilitate the rebuilding of the island \citep{Rodekruis2017}. \\

\noindent Around the world, in both wealthy and impoverished regions, individuals and organisations, both governmental and \ac{ngo}, are motivated to reduce and manage the impact of disasters \citep{Coppola2015}. Within this disaster management the decision making process and risk management process are bounded by three key characteristics \citep{Zlatanova2008}. [1.] Rapid action needs to be taken, [2.] aware of the situation and context [3.] with a connected overview of the data available. The goal of the \ac{drm} is to minimize the impact from a disaster \citep{Piero2012}. Information on the extent of the damage is therefore paramount, as demonstrated by relief operations from the \ac{nlrc}. Building damage is an essential indicator for this \citep{Schweier2006}; but can be hard to establish as it requires a lot of manual field labour; a dangerous and timely ordeal for aid workers involved \citep{Kerle2010}. Automated detection of building damages based on remotely sensed data could be the solution, allowing for faster and more efficient response \citep{Vetrivel2016b}. \\

\noindent Remote sensing has long been part of the \ac{drm} cycle. According to \citet{Kerle2015} this started at the beginning of space-based remote sensing around the 1960s and 1970s and brought about the increase in information within the \ac{drm}. From here the development of remote sensing techniques, both in space (optical and \ac{sar}) and within the atmosphere (aerial and \ac{uav} based),  accelerated over the past 50 years and increasingly allows for higher resolution information in a more timely manner. The performance increase in remote sensing solutions make it applicable for the automated classification of building damage \citep{DellAcqua2012,Dong2013}. Many solutions for automated damage detection or classification have been developed over the past years in academia, based on several remote sensing techniques. \citet{Dong2013} provides a clear overview of the solutions up until 2013 and several more have been developed since \citep{Dominici2017,Sharma2017,Kakooei2017,Vetrivel2016b,Menderes2015}. However, in practice, services from the International Charter, like Copernicus and UNOSAT, are mostly used for \ac{sem} \citep{Voigt2016} as they produce usable results \citep{Kerle2010}. The method for damage classification used by these services is manual visual interpretation of remotely sensed data, as is indicated by the disclaimers or map information of products from these services and a program specialist at UNOSAT \citep{Cop2017,UNDAC2017}. While this approach is safer for the aid workers in the field, it remains a laboursome task. 


\section{Problem statement}
\noindent It seems remarkable that the extensive academic research on automated damage detection is not implemented in the disaster relief sector, as the building damage is a fundamental indicator used in \ac{drm} and relief operations \citep{Schweier2006}. The lack of implementation of automated methods can be seen as an indicator of the absence of support from the humanitarian agencies. Several considerations could be the cause of this; [1.] there is too little communication between humanitarian agencies and academics, resulting in too complex methods or inadequate solutions. [2.] the methods proposed do not deliver the expected outcomes concerning effectiveness or accuracy. [3.] there are no resources to implement the new methods in existing procedures. An example of this can be found in \citet{Ajmar2011}. This paper mentions the lack of predictable results and time involved as impediments for implementation of automated approaches. Disaster situations require fast implementations as lives might be at stake, while reliable results are necessary for fair distribution of aid. \\

\noindent The stringent requirements from humanitarian organisations in disaster situations results in the recurrence of visual interpretation within damage detection from remotely sensed data. In this process a group of professionals has to visually inspect pre-event and post-event imagery to conclude the extent of damage to areas. Even though this is common practice, it is only viable on smaller areas, as the task of human comparison is labour intensive and requires specialist knowledge. The return to this method is aided by the understanding that automated approaches are not as accurate as human interpretation from satellite imagery. However, this is difficult to corroborate as academic literature seems to insinuate the opposite. \\

\noindent This research explores methods for the accurate classification of building damage after a natural disaster. Existing academic methods will be taken into consideration and tested on the available -real world- data from St. Maarten. Furthermore, the research will be conducted in cooperation with \ac{nlrc} to cope with some of considerations that might be causing the lack of implementation. Advances in remote sensing techniques, machine learning and \ac{gis} are recognised as upcoming and supportive technologies within the organisation, as it established a new data team [510] in 2016. The data team and \ac{nlrc}, as well as other humanitarian organisations, could benefit from the research into the automated classification of building damage after a disaster, as it would allow for more efficient delivery of aid and humanitarian relief. The academic field working on remote sensing for disaster situations could also benefit from this research as it will provide a comparison between methods in a scenario different from the academic examples.\\

\noindent An viable method would enable humanitarian organisations to accurately plan their interventions and relief operations on the basis of damage inflicted in a disaster struck areas. This would allow for an increase in effectiveness of the activities, and therefore help more people affected by a natural disaster.

\section{Research objectives} \label{sec:recobj}
The objective of this research is: \textit{to develop a method for the automatic classification of damage inflicted by hurricanes using remotely sensed data; applied to the case of hurricane Irma on the island of St. Maarten.} This will be based on already existing methods, which will be extended to suit the requirements from humanitarian organisations for improved aid delivery. To achieve this goal the following research question will be answered:
\begin{quote}
	\textit{Is the use of remotely sensed data a viable option for the automatic classification of \mbox{hurricane} inflicted damage?}
\end{quote}

\noindent To develop such method various other challenges will need to be handled. These will be dealt with in the literature research and analysis of existing academic methods. To guide this process the following sub-questions have been formulated:

\begin{itemize}
	\item How is damage determined?
	\item What criteria are set for damage classification methods?
	\item Which methods already exist?
	\item How do these methods perform?
	\item How does the state of the art compare to these methods?
\end{itemize}

\noindent With answers to these sub-questions, a description of an effective method is developed. This allows for insight in the necessities of such method. The difference between damage detection, classification, and assessment is also considered as all three require different approaches. The restrictions imposed by humanitarian organisations, especially within disaster situations, are also taken into consideration; as these might impose other criteria on methods. As described in this chapter various methods already exist, a subset of these will be considered in this research after the criteria for damage classification have been established. The next scientific challenge lies in the determination of the effectiveness of existing methods. The main question to be answered in this regard is the accuracy comparison between ground truth data from St. Maarten and the results from these methods. An assessment to classify accuracy is necessary to be able to compare the methods. To better map the impact of a disaster an extended overview of the damage is necessary, to achieve this the methods will be examined for extension to allow for damage classification. To ensure the possible solutions and extensions can be applied, these will be evaluated for performance compared to existing methods and ground truth. This will take into account the data derived from Copernicus assessment of the island from September 2017. With answers to all these sub-questions and the main research question, it would be necessary to see the methods and possible extensions on a broader scale and consider their fit within both the academic field and humanitarian field of operation. This will allow for reflection on the results and might allow for discussion on the possibility of implementation in the field.

\section{Research methodology} \label{sec:met}

In this section the methodology to achieve the main research goal will be defined. As shown in figure \ref{fig:met}, the methodology for this research is linearly structured with 3 sub-groups, Preparation, Research implementation, and assessments of Results. To answer the first three sub-questions an extensive literature research has been set-up as preparation for the implementation of existing methods. On this bases, methods have been selected to be implemented and possibly extended. Lastly, the results from all implementations will be compared and discussed.\\

\begin{figure}[!h]
	\centering
	\captionsetup{justification=raggedright,singlelinecheck=false}
	\includegraphics[width=0.50\textwidth]{figs/method1.png}
	\caption{\footnotesize{Methodology structure, highlighting the various phases [Grey = Preparation, Green = Research implementation, and Yellow = Assessment of Results]}}
	\label{fig:met}
\end{figure}

\subsection{Preparation} \label{ssec:prep}
The literature review as preparation for the implementation and comparison will form the base for this research. To allow for a thorough, yet workable, approach the following guiding principles have been used. The inclusion criteria for the research were different for the various topics. In general the literature are peer reviewed papers or masters or doctorate theses from the past 15 years. However, exceptions have been made for explanations of guiding principles. All literature is related to the topic they are describing and have been found through a combination of related key-words. The literature review regarding existing methods have been more strictly limited to the past 15 years with emphasis on newer material. Furthermore, have solutions for all disaster been considered, as long as the damage descriptions were not disaster specific. Particularly design studies have been regarded, in which new methods were developed. This process was guided by key-words and other literature comparing methods. The results from this review can be found in chapter \ref{framework}.

\noindent From the literature review a systematic approach to the testing of methods has been developed. This systematic approach considers both data and method in the comparison. A description of this approach is provided in chapter \ref{implem}.
\subsection{Implementation and Comparison}
The implementation of methods and the comparison of results is based on the systematic approach defined in the preparation based on the knowledge acquired from the literature review. A selection of methods has been implemented using the description of their respective literature; as well as extended to fit the requirements for classification in a disaster situation. The results from this implementation have been compared to the \ac{stoa} approaches in use in humanitarian response to disasters and ground truth data acquired. This comparison provides an insight in the viability of methods to be implemented in disaster situations and possible future research.

\section{Research scope} \label{sec:scope}
This research will focus on the accurate and automated classification of building damage in a hurricane struck area. Limitations in a masters thesis are unavoidable as time and resources are limited. Spatially the research is restricted to the Dutch part of the island St. Maarten as well as temporally limited to the aftermath of hurricane Irma in 2017. These limitations are set forward by the data available and is also supported by the fact that the author has been to the island in the aftermath of the hurricanes. This on-the-ground experience could be helpful in the understanding of the problem. \\
The method to be developed will be based on existing methods and will use, for where possible, existing software packages. This will shift the focus from a completely new method, to an adaptation with potential for implementation in various disaster procedures. This allows for more investigation into existing methods, and selecting effective techniques. By choosing this approach, developing a method which would be added to a list of unused, existing methods, is avoided. To further reduce the scope, only two existing methods will be selected for rigorous assessment. These will be chosen through the theoretical framework and literature research, set forward in chapter \ref{framework}. \\
For this research, the task of object detection, as used by various other academic methods \citep{Vetrivel2016b,Kakooei2017}, will not be used. The datasets available allow for the use of existing building outlines, gathered from recent data. The \ac{nlrc} uses voluntary cartographers to map disaster areas shortly after an incident and where possible, in case of the hurricane on St. Maarten, shortly before a disaster. Making sure that maps are up to date and can be used for planning. This eliminates the need for object detection from data sources, however requires the data input to be properly matched.\\
The research is limited to the various datasets available through the 510 team of the \ac{nlrc} as cleaning, relief and reconstruction activities on the island make new data collection impossible. However, these datasets can be considered a good representation of available data in the wake of large scale disaster. The cooperation with the data team of the \ac{nlrc} allows for a good balance between the technical approach of the Delft University of Technology and a more societal aspect from a humanitarian organisation.\\ 

\section{Reading guide}
The following will be discussed in this document:

\begin{itemize}
	\item Chapter \ref{framework} will go into further detail concerning the theoretical framework of this research. Existing methods will be inventoried and two will be selected for further examination. More details on background topics concerning these methods and the methodology will be provided as well.
	\item Chapter \ref{implem} describes the implementation of the existing methods and the extension to allow for classification over detection.
	\item Chapter \ref{result} showcases the results from the various implementations. All of these are supported by a short discussion with regards to the various requirements and accuracy measures.
	\item Chapter \ref{concl} concludes the research and summarises the findings.
	\item Chapter \ref{reco} lists possible relevant new approaches or angles to problems encountered within the analysis, as well as areas of research which could result in new solutions or findings.
\end{itemize}


