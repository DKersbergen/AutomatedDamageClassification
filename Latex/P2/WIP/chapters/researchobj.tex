\chapter{Research objectives} \label{research}

The objective of this research is: \textit{to find a method for the automatic classification of damage inflicted by hurricanes on the island of St. Maarten using remotely sensed data.} Based on the findings of \citet{Kakooei2017} the research will be focused around an implementation from various sources for improvement in the accuracy of the classification To achieve this the following goals have been derived:

\begin{itemize}
	\item Compare existing methods for building damage classification to each other and ground truth data.
	\item Find complementary methods and aspects of data that allow for fusion of data and methods to achieve an increase in accuracy.
	\item Develop a method which fuses the various data sources and methods to facilitate accurate classification of damage to buildings. 
	\item Compare the results from the proposed method to existing methods and ground truth data.
\end{itemize}

With results from these goals the research question will be answered: \textit{Is the fusion of remotely sensed data a viable option for the automatic classification of hurricane inflicted damage?}\\

An viable method would enable humanitarian organisations to accurately plan their interventions and relief operations on the basis of damage inflicted in disaster struck areas. This would allow for an increase in effectiveness of the activities, and therefore help more people affected by a natural disaster. To develop such method various other challenges will need to be handled. These will be dealt with in the literature research and analysis of existing academic methods.\\ 
A description of an effective method is necessary to develop one. To gain insight in the needs and possibilities of methods the following question will need to be answered: Which criteria are most useful for a damage classification framework? With which in turn two other questions are formed, what is damage classification and how does it differ from damage detection? And humanitarian organisations and disaster situations might impose various other criteria on methods. which should therefore be considered in the description as well.\\
The next scientific challenge lies in the determination of the effectiveness of existing methods. The main question to be answered in this regard is the accuracy comparison between ground truth data from St. Maarten and the results from these methods. The fusion of data is expected to improve the accuracy of the various methods, which subsequently will be one of the main indicators within the classification framework. From these results it will be possible to select the aspects of methods and data which will allow for fusion and also result in an accuracy boost in the new method.\\
From the answers to these challenge the new method can be developed but will need to be validated to ensure the results are confirming to expectation and suffice for the use in humanitarian context. To do so the new results will need to be compared to the benchmark results from the existing methods. \\

With answers to all these challenges and the main research question, it would be necessary to see the proposed method on a broader scale and the fit within both the academic field and humanitarian field of operation. This will allow for reflection on the results and might allow for discussion on the possibility of implementation in the field.

\section{research scope}
This research will focus on the accurate and automated classification of building damage in a hurricane struck area. Limitations in a masters thesis are unavoidable as time and resources are limited. Spatially the research is restricted to the Dutch part of the island St. Maarten as well as temporally limited to the aftermath of hurricane Irma in 2017. This limitation is also supported by the fact that the author has been to the island in the aftermath of the hurricanes. This on-the-ground experience could be helpful in the understanding of the problem. \\
The method to be developed will be based on existing methods and will use, for where possible, existing software packages. This will shift the focus from a completely new method, to an adaptation with potential for implementation in various disaster procedures. This allows for more investigation into existing methods, and cherry-picking effective techniques. By choosing approach developing a method which would be added to a list of unused, existing methods, is avoided. To further reduce the scope, only two existing methods will be selected for rigorous assessment. These will be chosen through the theoretical framework and literature research. \\
For this research, the task of object detection, as used by various other academic methods \citep{Vetrivel2016b,Kakooei2017}, will not be used. The datasets available allow for the use of existing building outlines, gathered from recent data. The \ac{nlrc} uses voluntary cartographers to map disaster areas shortly after a disaster and where possible, in case of the hurricane on St. Maarten, shortly before a disaster. Making sure that maps are up to date and can be used for planning. This eliminates the need for object detection from data sources, however requires the data input to be properly matched.\\
The research is limited to the various datasets available through the 510 team of the \ac{nlrc} as cleaning, relief and reconstruction activities on the island make new data collection impossible. However these datasets can be considered a good representation of available data in the wake of large scale disaster. The cooperation with the data team of the \ac{nlrc} allows for a good balance between the technical approach of the Delft University of Technology and a more societal aspect from a humanitarian organisation.\\ 
Section \ref{sec:def} will elaborate some of the scope reductions with literature research.
